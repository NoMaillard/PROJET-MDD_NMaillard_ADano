\documentclass[a4paper,11pt]{article}
\usepackage[top=3.5cm, bottom=2.5cm, left=3cm, right=3cm]{geometry}
\usepackage[T1]{fontenc}
\usepackage[utf8]{inputenc}
\usepackage[frenchb]{babel}
\usepackage{ifpdf}
\usepackage{asymptote}
\usepackage{graphicx}
\usepackage{fancybox}
\usepackage{parallel}
\usepackage{amsmath}
\usepackage{tikz}
\usepackage{amssymb}
\usepackage{enumitem}
\usepackage{wallpaper}

\usetikzlibrary{arrows,shapes,positioning,shadows,trees,automata}
\newcommand{\gamename}{Spazz }

\tikzset{
    scenario/.style={
        	rectangle,
           	rounded corners,
           	draw=black, very thick,
           	fill=green!30,
           	minimum height=3em,
           	inner sep=2pt,
           	text centered,
           	drop shadow
           },
    action/.style={
    		ellipse,
    		minimum height=1.2cm,
    		draw=black,very thick,
    		fill=red!60,
    		text centered,
    		drop shadow
    		},
    }


\begin{document}
\URCornerWallPaper{.25}{/Users/Noe/Documents/Cours/S1/Ressources/Logo.png}
\begin{titlepage}

\newcommand{\HRule}{\rule{\linewidth}{0.5mm}} % Defines a new command for the horizontal lines, change thickness here

\center % Center everything on the page
 
%----------------------------------------------------------------------------------------
%	HEADING SECTIONS
%----------------------------------------------------------------------------------------

\textsc{\LARGE École Nationale des Ingénieurs de Brest}\\[1.5cm]
\textsc{\Large Cahier des Charges}\\[0.5cm]
\textsc{\large MDD-PROJET}\\[0.5cm]

%----------------------------------------------------------------------------------------
%	TITLE SECTION
%----------------------------------------------------------------------------------------

\HRule \\[0.4cm]
{\huge \bfseries Spazz}\\[0.4cm] % Title of your document
\HRule \\[1.5cm]
 
\Large
Noé \textsc{Maillard} et Allan \textsc{Dano}\\[3cm]


{\large 27 mars 2014}\\[3cm]

{\large Version 1.0}
 

\vfill

\end{titlepage}
\newpage
\tableofcontents
\newpage
\section{Objectifs}

\subsection{Description générale}


Dans le cadre du cours de MDD, nous avons eu l'idée de créer un jeu vidéo mélangeant le concept du Snake et celui de jeu par niveau, à la manière de Mario.

\section{Expression du besoin}

\subsection{Règles du jeu}
\begin{itemize}[label = $\bullet$]
	\item l'utilisateur déplace une entité nommée Spazz qui doit se rendre à la fin du niveau avant la fin du temps imparti
	\item Avant de pouvoir terminer un niveau, Spazz doit ramasser un nombre de jetons
	\item Les jetons peuvent prendre différentes valeurs
	\item À chaque fois que Spazz ramasse un jeton, il grandit en taille
	\item À la fin d'un niveau, le nombre de points est la somme du temps restant au compteur et des points des jetons ramassés
\end{itemize}

Le but du jeu est de terminer tous les niveaux.

\subsection{L'interface}

\subsubsection{Visuel}

Pendant une partie l'utilisateur voit le niveau en cours avec Spazz qui se déplace, au dessus il y a une banderole qui affiche le score total, le temps restant et les nombre de jetons restants.

\subsubsection{Interaction}

L'utilisateur agit sur Spazz à l'aide l'aide des touches Z , Q , S et D ou des touches directionnelles. Ces touches changent la direction dans laquelle Spazz se dirige.


\subsection{Manuel Utilisateur}

\begin{description}
	\item [Lancer le Jeu : ] \texttt{\$python Spazz}
	\item [Choisir le pseudo : ] entrer le pseudo confirmer avec la touche entrée.
	\item [Choisir la difficulté : ] entrer la difficulté choisie
\end{description}

\subsection{Contraintes techniques}

\begin{itemize}[label = $\bullet$]
	\item Le logiciel crée est évalué par les professeurs sur un ordinateur de salle de TP, il faut donc que le jeu s’exécute et soit jouable sur ces machines
	\item Le cours porte sur le langage Python, il est donc évident que le jeu soit écrit en Python
	\item Le paradigme utilisé est celui de la programmation procédurale
	\item L'interface doit être en mode texte dans le terminal
\end{itemize}

\subsection{Scénario d'utilisation}
\noindent
\textbf{S0 : Scénario du menu}\\[.5cm]
\begin{tikzpicture}[->,>=stealth']

	 \node[scenario] (Lancer) 
		{
			Lancer le Jeu
		};
	  
	 \node[action,
	 	below of=Lancer,
	 	xshift = 1.5 cm,
	 	node distance = 2.5 cm
	 	]
	 	(Pseudo)
		{
			Choisir un pseudo
		};
		 
	 \node[action,
		right of=Pseudo,
		yshift = 2cm,
		node distance=3.5cm
		]
		(Difficult) 
		{
		 	Choisir la difficulté
		};

	 \node[scenario,
		left of=Lancer,
		node distance=3.5cm,
		]
		(Niveau) 
		{
			S1 : Jouer un niveau
		};

	 \path 	(Lancer)	edge[bend left = 10] (Pseudo)
	 		(Lancer)	edge[bend left = 10] (Difficult)
	 		(Lancer)	edge[bend right] (Niveau);
\end{tikzpicture}

\noindent
\textbf{S1 : Jouer un niveau} \\[.5cm]

\begin{tikzpicture}[->,>=stealth']
	
	\node[scenario,
		text width = 3cm]
		(Niveau) 
		{
			S1 : Jouer un niveau
		};

	\node[action,
		right of = Niveau,
		xshift = 1cm,
		node distance = 3.5cm,
		]
		(Choix)
		{
			Choisir le niveau
		};

	\node[action,
		text width = 3cm,
		below of = Niveau,
		node distance = 3.5cm
		]
		(Afficher)
		{
			Afficher le niveau
		};

	\node[action,
		text width = 3cm,
		yshift = 1.5cm,
		right of = Afficher,
		node distance = 3.5cm
		]
		(Direction)
		{
			Changer de direction
		};

	\node[action,
		text width = 3cm,
		yshift = 1.5cm,
		left of = Afficher,
		node distance = 3.5cm
		]
		(Jeton)
		{
			Ramasser un jeton
		};

	\node[action,
		left of = Jeton,
		yshift = 2cm,
		node distance = 3.5cm,
		text width = 3cm
		]
		(Finir)
		{
			Finir le niveau
		};

	\node[action,
		below of = Finir,
		node distance = 4cm,
		text width = 3cm
		]
		(Meilleurs)
		{
			Afficher high scores
		};

	\path 	(Niveau)	edge[bend left = 10] (Finir)
			(Niveau)	edge[bend right = 20] (Choix)
	 		(Niveau)	edge[bend left = 20] (Jeton)
	 		(Niveau)	edge[bend right = 20] (Direction)
	 		(Niveau)	edge[bend right = 5](Afficher)
	 		(Finir) 	edge[bend right = 30](Meilleurs);
\end{tikzpicture}


\newpage
\section{Analyse du besoin}
	
\subsection{Fonctionalités}

\begin{enumerate}[label*= F\arabic*,font = \textbf]
	\item : Choisir un pseudo
	\item : Choisir la difficulté
	\item : Jouer un niveau
	\begin{enumerate}[label*=.\arabic*,font = \textbf]
		\item : Choisir le niveau
		\item : Afficher le Jeu
		\item : Changer de direction
		\item : Ramasser un jeton
		\item : Finir le niveau
		\begin{enumerate}[label*=.\arabic*,font = \textbf]
			\item : Afficher résultat
			\item : Afficher les meilleurs scores
			\item : Quitter le jeu
		\end{enumerate}
	\end{enumerate}
\end{enumerate}
\section{Critères de validité et de qualité}

\subsection{Validation}


La validation du logiciel sera validée de la manière suivante :\\
\begin{itemize}[label = $\bullet$]
	\item Le code doit s’exécuter correctement en suivant les instructions livrées avec le logiciel.
	\item L'utilisation du logiciel permettra de constater que les fonctionnalités ont bien été implémentées.
\end{itemize}

\subsection{Qualité}


Différent critères permettront d'évaluer la qualité du jeu:\\
\begin{itemize}[label = $\bullet$]
	\item La jouabilité:L'interface devra être suffisamment ergonomique pour permettre au joueur de jouer un niveau sans difficultés techniques (Lenteur, bugs).
	\item La robustesse.
	\item Le respect de méthodes de codage données en cours de MDD.
\end{itemize}

\section{Livrables} 

\subsection{Echéancier}

\noindent
Avant la 3\textsuperscript{ème} UC : Prototype P1 \\
Avant la 5\textsuperscript{ème} UC : Prototype P2 \\
Avant la Dernière UC: Version Finale 
\subsection{Description des livrables}

\begin{description}
	\item [Cahier des Charges]	Expression et analyse du besoin.\\
								Fichier : \texttt{Cahier\_des\_charges.pdf}
	\item [Document de conception] Fichier : \texttt{Document\_de\_conception-vx.x.pdf}
	\item [Prototype P1] 	Ce prototype porte sur la création et l'affichage d'un niveau.\\
							Mise en œuvre de fonctionnalités : F1, F2, F3.2, F3.2\\
							Fichier : \texttt{Spazz-P1.tgz}
	\item [Prototype P2]
	\item [Version finale] Ficher : \texttt{Spazz-v1.0.tgz}
\end{description}


\end{document}