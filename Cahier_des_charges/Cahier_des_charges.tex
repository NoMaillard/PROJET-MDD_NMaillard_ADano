\documentclass[a4paper,11pt]{scrartcl}
\usepackage[top=3.5cm, bottom=2.5cm, left=3cm, right=3cm]{geometry}
\usepackage[T1]{fontenc}
\usepackage[utf8]{inputenc}
\usepackage[frenchb]{babel}
\usepackage{ifpdf}
\usepackage{asymptote}
\usepackage{graphicx}
\usepackage{fancybox}
\usepackage{parallel}
\usepackage{amsmath}
\usepackage{tikz}
\usepackage{amssymb}
\usepackage{enumitem}
\usepackage{wallpaper}

\usetikzlibrary{arrows,shapes,positioning,shadows,trees,automata}
\newcommand{\gamename}{Spazz }

\tikzset{
    scenario/.style={
        	rectangle,
           	rounded corners,
           	draw=black, very thick,
           	fill=green!30,
           	minimum height=3em,
           	inner sep=2pt,
           	text centered,
           	drop shadow
           },
    action/.style={
    		ellipse,
    		minimum height=1.2cm,
    		draw=black,very thick,
    		fill=red!60,
    		text centered,
    		drop shadow
    		},
}


\begin{document}
\URCornerWallPaper{.25}{/Users/Noe/Documents/Cours/S1/Ressources/Logo.png}
\begin{titlepage}

\newcommand{\HRule}{\rule{\linewidth}{0.5mm}} % Defines a new command for the horizontal lines, change thickness here

\center % Center everything on the page

%----------------------------------------------------------------------------------------
%	HEADING SECTIONS
%----------------------------------------------------------------------------------------

\textsc{\LARGE École Nationale des Ingénieurs de Brest}\\[1.5cm]
\textsc{\Large Cahier des Charges}\\[0.5cm]
\textsc{\large MDD-PROJET}\\[0.5cm]

%----------------------------------------------------------------------------------------
%	TITLE SECTION
%----------------------------------------------------------------------------------------

\HRule \\[0.4cm]
{\huge \bfseries Spazz}\\[0.4cm] % Title of your document
\HRule \\[1.5cm]

\Large
Noé \textsc{Maillard} et Allan \textsc{Dano}\\[3cm]


{\large \today}\\[3cm]

{\large Version 1.1}


\vfill

\end{titlepage}
\newpage
\tableofcontents
\newpage
\section{Objectifs}

\subsection{Description générale}


Dans le cadre du cours de MDD, nous avons eu l'idée de créer un jeu vidéo mélangeant le concept du Snake et celui de jeu par niveau.

\section{Expression du besoin}

\subsection{Règles du jeu}
\begin{itemize}[label = $\bullet$]
	\item l'utilisateur déplace une entité nommée Spazz qui doit se déplacer dans le niveau en ramassant des jetons
	\item À chaque fois que Spazz ramasse un jeton, il grandit en taille
	\item À la fin d'un niveau, le nombre de points est le nombre de jetons ramassés par Spazz.
    \item le niveau est terminé lorsque Spazz se cogne contre un mur ou lui-même.
\end{itemize}

Le but du jeu est de manger le plus de jetons dans un niveau

\subsection{L'interface}

\subsubsection{Visuel}
Dans le niveau, l'utilisateur voit les différents items du menu principal et une flèche qui indique quel item est sélectionné.
Pendant une partie l'utilisateur voit le niveau en cours avec Spazz qui se déplace en fonction des touches appuyées, en dessous il y a une banderole qui affiche le score total, le temps restant et les nombre de jetons restants.
Dans l'éditeur de niveaux, l'utilisateur voit le niveau, le curseur et deux nombres indiquant la position de son curseur et le mode dans le quel il se trouve.

\subsubsection{Interaction}
Dans le menu, lorsqu'il y a plusieurs items, l'utilisateur navigue à l'aide des touches Z et S puis confirme par la touche E, lorsque l'utilisateur entre du texte, il confirme avec la touche entrée.
Pendant une partie, l'utilisateur agit sur le Spazz l'aide des touches Z , Q , S et D. Ces touches changent la direction dans laquelle Spazz se dirige.
Dans l'éditeur de niveaux, l'utilisateur déplace le curseur à l'aide des touche Z, Q, S et D. en appuyant sur la barre d'espace il change le mode du curseur.


\subsection{Manuel Utilisateur}

\begin{description}
    \item [Lancer le Jeu : ] \texttt{\$python Spazz}
	\item [Choisir le nom : ] dans le menu, sélectionner l'item "Change name", entrer le nom choisi et confirmer.
    \item [Choisir la difficulté : ] dans le menu, sélectionner l'item "Change Difficulty", et entrer 1, 2 ou 3 en fonction de la difficulté désirée.
    \item [Choisir le niveau : ] dans le menu, sélectionner l'item "Select level", entrer le numéro du niveau désiré puis confirmer.
    \item [Jouer le niveau : ] dans le menu, sélectionner l'item "play".
    \item [Quitter le Jeu : ] dans le menu, sélectionner l'item "quit game".
    \item [Diriger le Spazz : ] dans le jeu, appuyer sur les touches Z, Q, S, ou D pour changer la direction dans laquelle le Spazz se dirige
    \item [Créer un niveau : ] dans l'éditeur de niveau, appuyer sur les touches Z, Q, S ou D pour déplacer le curseur, appuyer sur la barre d'espace pour commencer a placer des murs ou effacer des murs déjà placés et ré-appuyer sur la barre d'espace pour seulement déplacer le curseur. Appuyer sur la touche entrée pour sauvegarder le niveau et revenir au menu.
\end{description}

\subsection{Contraintes techniques}

\begin{itemize}[label = $\bullet$]
	\item Le logiciel crée est évalué par les professeurs sur un ordinateur de salle de TP, il faut donc que le jeu s’exécute et soit jouable sur ces machines
	\item Le cours porte sur le langage Python, il est donc évident que le jeu soit écrit en Python
	\item Le paradigme utilisé est celui de la programmation procédurale
	\item L'interface doit être en mode texte dans le terminal
\end{itemize}

\subsection{Scénario d'utilisation}
\noindent
\textbf{S0 : Scénario du menu}\\[.5cm]
\begin{tikzpicture}[->,>=stealth']

	 \node[scenario] (Lancer)
		{
			Lancer le Jeu
		};

	 \node[action,
	 	below of=Lancer,
	 	xshift = 1.5 cm,
	 	node distance = 2.5 cm
	 	]
	 	(Pseudo)
		{
			Choisir un pseudo
		};

	 \node[action,
		right of=Pseudo,
		yshift = 2cm,
		node distance=3.5cm
		]
		(Difficulty)
		{
		 	Choisir la difficulté
		};

	 \node[scenario,
		left of=Lancer,
		node distance=3.5cm,
        yshift = 1cm
		]
		(Niveau)
		{
			S1 : Jouer un niveau
		};

    \node[action,
        above of = Difficulty,
        node distance = 2.5cm,
        ]
        (Choose)
        {
            Choisir le niveau
        };

    \node[action,
        above of = Lancer,
        node distance = 2.5cm,
        ]
        (Quit)
        {
            Quitter le Jeu
        };
    \node[action,
        node distance = 3.5cm,
        left of = Lancer,
        yshift = -1cm
        ]
        (Edit)
        {
            Créer un Niveau
        };

	 \path 	(Lancer)	edge[bend left = 10] (Pseudo)
	 		(Lancer)	edge[bend left = 10] (Difficulty)
	 		(Lancer)	edge[bend right] (Niveau)
            (Lancer)    edge[bend left] (Choose)
            (Lancer)    edge[bend right] (Edit)
            (Lancer)    edge[bend left] (Quit);
\end{tikzpicture}

\noindent
\textbf{S1 : Jouer un niveau} \\[.5cm]

\begin{tikzpicture}[->,>=stealth']

	\node[scenario,
		text width = 3cm]
		(Niveau)
		{
			S1 : Jouer un niveau
		};

	\node[action,
		text width = 3cm,
		below of = Niveau,
		node distance = 3.5cm
		]
		(Afficher)
		{
			Afficher le niveau
		};

	\node[action,
		text width = 3cm,
		yshift = 1.5cm,
		right of = Afficher,
		node distance = 3.5cm
		]
		(Direction)
		{
			Changer de direction
		};

	\node[action,
		text width = 3cm,
		yshift = 1.5cm,
		left of = Afficher,
		node distance = 3.5cm
		]
		(Jeton)
		{
			Ramasser un jeton
		};

	\node[action,
		left of = Jeton,
		yshift = 2cm,
		node distance = 3.5cm,
		text width = 3cm
		]
		(Finir)
		{
			Finir le niveau
		};

	\node[scenario,
		below of = Finir,
		node distance = 4cm,
		text width = 3cm
		]
		(Menu)
		{
			Retourner au menu
		};

	\path 	(Niveau)	edge[bend left = 10] (Finir)
	 		(Niveau)	edge[bend left = 20] (Jeton)
	 		(Niveau)	edge[bend right = 20] (Direction)
	 		(Niveau)	edge[bend right = 5](Afficher)
	 		(Finir) 	edge[bend right = 30](Menu);
\end{tikzpicture}


\newpage
\section{Analyse du besoin}

\subsection{Fonctionalités}

\begin{enumerate}[label*= F\arabic*,font = \textbf]
	\item : Choisir un pseudo
	\item : Choisir la difficulté
    \item : Choisir le niveau
	\item : Jouer un niveau
    \begin{enumerate}[label*=.\arabic*,font = \textbf]
		\item : Afficher le Jeu
		\item : Changer de direction
		\item : Ramasser un jeton
		\item : Finir le niveau
    \end{enumerate}
	\item : Afficher les meilleurs scores
	\item : Quitter le jeu
    \item : Créer des Niveaux
\end{enumerate}
\section{Critères de validité et de qualité}

\subsection{Validation}


La validation du logiciel sera validée de la manière suivante :\\
\begin{itemize}[label = $\bullet$]
	\item Le code doit s’exécuter correctement en suivant les instructions livrées avec le logiciel.
	\item L'utilisation du logiciel permettra de constater que les fonctionnalités ont bien été implémentées.
\end{itemize}

\subsection{Qualité}


Différent critères permettront d'évaluer la qualité du jeu:\\
\begin{itemize}[label = $\bullet$]
	\item La jouabilité : L'interface devra être suffisamment ergonomique pour permettre au joueur de jouer un niveau sans difficultés techniques (Lenteur, bugs).
	\item La robustesse.
	\item Le respect de méthodes de codage données en cours de MDD.
\end{itemize}

\section{Livrables}

\subsection{Echéancier}

\noindent
Avant la 3\textsuperscript{ème} UC : Prototype P1 \\
Avant la 5\textsuperscript{ème} UC : Prototype P2 \\
Avant la Dernière UC: Version Finale
\subsection{Description des livrables}

\begin{description}
	\item [Cahier des Charges]	Expression et analyse du besoin.\\
								Fichier : \texttt{Cahier\_des\_charges.pdf}
	\item [Document de conception] Fichier : \texttt{Document\_de\_conception-vx.x.pdf}
    \item [Code Source]  Fichiers contenus dans le dossier \texttt{src}
	\item [Prototype P1] 	Ce prototype porte sur la création et l'affichage d'un niveau.\\
							Mise en œuvre de fonctionnalités : F1, F2, F3.2, F3.2\\
							Fichier : \texttt{Spazz-P1.tgz}
	\item [Version finale] Ficher : \texttt{Spazz-v1.0.tgz}
\end{description}


\end{document}